% Options for packages loaded elsewhere
\PassOptionsToPackage{unicode}{hyperref}
\PassOptionsToPackage{hyphens}{url}
%
\documentclass[
]{article}
\usepackage{amsmath,amssymb}
\usepackage{iftex}
\ifPDFTeX
  \usepackage[T1]{fontenc}
  \usepackage[utf8]{inputenc}
  \usepackage{textcomp} % provide euro and other symbols
\else % if luatex or xetex
  \usepackage{unicode-math} % this also loads fontspec
  \defaultfontfeatures{Scale=MatchLowercase}
  \defaultfontfeatures[\rmfamily]{Ligatures=TeX,Scale=1}
\fi
\usepackage{lmodern}
\ifPDFTeX\else
  % xetex/luatex font selection
\fi
% Use upquote if available, for straight quotes in verbatim environments
\IfFileExists{upquote.sty}{\usepackage{upquote}}{}
\IfFileExists{microtype.sty}{% use microtype if available
  \usepackage[]{microtype}
  \UseMicrotypeSet[protrusion]{basicmath} % disable protrusion for tt fonts
}{}
\makeatletter
\@ifundefined{KOMAClassName}{% if non-KOMA class
  \IfFileExists{parskip.sty}{%
    \usepackage{parskip}
  }{% else
    \setlength{\parindent}{0pt}
    \setlength{\parskip}{6pt plus 2pt minus 1pt}}
}{% if KOMA class
  \KOMAoptions{parskip=half}}
\makeatother
\usepackage{xcolor}
\usepackage[margin=1in]{geometry}
\usepackage{color}
\usepackage{fancyvrb}
\newcommand{\VerbBar}{|}
\newcommand{\VERB}{\Verb[commandchars=\\\{\}]}
\DefineVerbatimEnvironment{Highlighting}{Verbatim}{commandchars=\\\{\}}
% Add ',fontsize=\small' for more characters per line
\usepackage{framed}
\definecolor{shadecolor}{RGB}{248,248,248}
\newenvironment{Shaded}{\begin{snugshade}}{\end{snugshade}}
\newcommand{\AlertTok}[1]{\textcolor[rgb]{0.94,0.16,0.16}{#1}}
\newcommand{\AnnotationTok}[1]{\textcolor[rgb]{0.56,0.35,0.01}{\textbf{\textit{#1}}}}
\newcommand{\AttributeTok}[1]{\textcolor[rgb]{0.13,0.29,0.53}{#1}}
\newcommand{\BaseNTok}[1]{\textcolor[rgb]{0.00,0.00,0.81}{#1}}
\newcommand{\BuiltInTok}[1]{#1}
\newcommand{\CharTok}[1]{\textcolor[rgb]{0.31,0.60,0.02}{#1}}
\newcommand{\CommentTok}[1]{\textcolor[rgb]{0.56,0.35,0.01}{\textit{#1}}}
\newcommand{\CommentVarTok}[1]{\textcolor[rgb]{0.56,0.35,0.01}{\textbf{\textit{#1}}}}
\newcommand{\ConstantTok}[1]{\textcolor[rgb]{0.56,0.35,0.01}{#1}}
\newcommand{\ControlFlowTok}[1]{\textcolor[rgb]{0.13,0.29,0.53}{\textbf{#1}}}
\newcommand{\DataTypeTok}[1]{\textcolor[rgb]{0.13,0.29,0.53}{#1}}
\newcommand{\DecValTok}[1]{\textcolor[rgb]{0.00,0.00,0.81}{#1}}
\newcommand{\DocumentationTok}[1]{\textcolor[rgb]{0.56,0.35,0.01}{\textbf{\textit{#1}}}}
\newcommand{\ErrorTok}[1]{\textcolor[rgb]{0.64,0.00,0.00}{\textbf{#1}}}
\newcommand{\ExtensionTok}[1]{#1}
\newcommand{\FloatTok}[1]{\textcolor[rgb]{0.00,0.00,0.81}{#1}}
\newcommand{\FunctionTok}[1]{\textcolor[rgb]{0.13,0.29,0.53}{\textbf{#1}}}
\newcommand{\ImportTok}[1]{#1}
\newcommand{\InformationTok}[1]{\textcolor[rgb]{0.56,0.35,0.01}{\textbf{\textit{#1}}}}
\newcommand{\KeywordTok}[1]{\textcolor[rgb]{0.13,0.29,0.53}{\textbf{#1}}}
\newcommand{\NormalTok}[1]{#1}
\newcommand{\OperatorTok}[1]{\textcolor[rgb]{0.81,0.36,0.00}{\textbf{#1}}}
\newcommand{\OtherTok}[1]{\textcolor[rgb]{0.56,0.35,0.01}{#1}}
\newcommand{\PreprocessorTok}[1]{\textcolor[rgb]{0.56,0.35,0.01}{\textit{#1}}}
\newcommand{\RegionMarkerTok}[1]{#1}
\newcommand{\SpecialCharTok}[1]{\textcolor[rgb]{0.81,0.36,0.00}{\textbf{#1}}}
\newcommand{\SpecialStringTok}[1]{\textcolor[rgb]{0.31,0.60,0.02}{#1}}
\newcommand{\StringTok}[1]{\textcolor[rgb]{0.31,0.60,0.02}{#1}}
\newcommand{\VariableTok}[1]{\textcolor[rgb]{0.00,0.00,0.00}{#1}}
\newcommand{\VerbatimStringTok}[1]{\textcolor[rgb]{0.31,0.60,0.02}{#1}}
\newcommand{\WarningTok}[1]{\textcolor[rgb]{0.56,0.35,0.01}{\textbf{\textit{#1}}}}
\usepackage{graphicx}
\makeatletter
\def\maxwidth{\ifdim\Gin@nat@width>\linewidth\linewidth\else\Gin@nat@width\fi}
\def\maxheight{\ifdim\Gin@nat@height>\textheight\textheight\else\Gin@nat@height\fi}
\makeatother
% Scale images if necessary, so that they will not overflow the page
% margins by default, and it is still possible to overwrite the defaults
% using explicit options in \includegraphics[width, height, ...]{}
\setkeys{Gin}{width=\maxwidth,height=\maxheight,keepaspectratio}
% Set default figure placement to htbp
\makeatletter
\def\fps@figure{htbp}
\makeatother
\setlength{\emergencystretch}{3em} % prevent overfull lines
\providecommand{\tightlist}{%
  \setlength{\itemsep}{0pt}\setlength{\parskip}{0pt}}
\setcounter{secnumdepth}{-\maxdimen} % remove section numbering
\ifLuaTeX
  \usepackage{selnolig}  % disable illegal ligatures
\fi
\usepackage{bookmark}
\IfFileExists{xurl.sty}{\usepackage{xurl}}{} % add URL line breaks if available
\urlstyle{same}
\hypersetup{
  pdftitle={Statistical Inference Course Project - Exponential Distribution and the CLT},
  pdfauthor={Felix Lopez},
  hidelinks,
  pdfcreator={LaTeX via pandoc}}

\title{Statistical Inference Course Project - Exponential Distribution
and the CLT}
\author{Felix Lopez}
\date{2024-08-01}

\begin{document}
\maketitle

\subsection{Introduction}\label{introduction}

In this project, I investigate the exponential distribution and compare
it with the Central Limit Theorem (CLT). The exponential distribution
can be simulated in R with \texttt{rexp(n,\ lambda)} where
\texttt{lambda} is the rate parameter. The mean of the exponential
distribution is \texttt{1/lambda} and the standard deviation is also
\texttt{1/lambda}. We will set \texttt{lambda\ =\ 0.2} for all of the
simulations. We will investigate the distribution of averages of 40
exponentials and perform a thousand simulations.

\subsection{Simulation}\label{simulation}

Simulating Exponential Distribution We will simulate the exponential
distribution and calculate the mean of 40 exponentials for each of the
1000 simulations.

\begin{Shaded}
\begin{Highlighting}[]
\CommentTok{\# Simulate 1000 averages of 40 exponentials}
\NormalTok{simulated\_means }\OtherTok{\textless{}{-}} \FunctionTok{replicate}\NormalTok{(simulations, }\FunctionTok{mean}\NormalTok{(}\FunctionTok{rexp}\NormalTok{(n, lambda)))}

\CommentTok{\# Theoretical mean and standard deviation}
\NormalTok{theoretical\_mean }\OtherTok{\textless{}{-}} \DecValTok{1} \SpecialCharTok{/}\NormalTok{ lambda}
\NormalTok{theoretical\_sd }\OtherTok{\textless{}{-}} \DecValTok{1} \SpecialCharTok{/}\NormalTok{ lambda }\SpecialCharTok{/} \FunctionTok{sqrt}\NormalTok{(n)}

\CommentTok{\# Sample mean and standard deviation}
\NormalTok{sample\_mean }\OtherTok{\textless{}{-}} \FunctionTok{mean}\NormalTok{(simulated\_means)}
\NormalTok{sample\_sd }\OtherTok{\textless{}{-}} \FunctionTok{sd}\NormalTok{(simulated\_means)}
\end{Highlighting}
\end{Shaded}

\begin{Shaded}
\begin{Highlighting}[]
\CommentTok{\# Results}

\NormalTok{sample\_mean}
\end{Highlighting}
\end{Shaded}

\begin{verbatim}
## [1] 5.011911
\end{verbatim}

\begin{Shaded}
\begin{Highlighting}[]
\NormalTok{theoretical\_mean}
\end{Highlighting}
\end{Shaded}

\begin{verbatim}
## [1] 5
\end{verbatim}

\begin{Shaded}
\begin{Highlighting}[]
\NormalTok{sample\_sd}
\end{Highlighting}
\end{Shaded}

\begin{verbatim}
## [1] 0.7749147
\end{verbatim}

\begin{Shaded}
\begin{Highlighting}[]
\NormalTok{theoretical\_sd}
\end{Highlighting}
\end{Shaded}

\begin{verbatim}
## [1] 0.7905694
\end{verbatim}

\#\#\#Comparing Sample Mean to Theoretical Mean

The sample mean of the simulated data is , while the theoretical mean is
r theoretical\_mean. As we can see, the sample mean is very close to the
theoretical mean, which is expected due to the Law of Large Numbers

Title (give an appropriate title) and Author Name

Overview: In a few (2-3) sentences explain what is going to be reported
on.

Simulations: Include English explanations of the simulations you ran,
with the accompanying R code. Your explanations should make clear what
the R code accomplishes.

Sample Mean versus Theoretical Mean: Include figures with titles. In the
figures, highlight the means you are comparing. Include text that
explains the figures and what is shown on them, and provides appropriate
numbers.

Sample Variance versus Theoretical Variance: Include figures (output
from R) with titles. Highlight the variances you are comparing. Include
text that explains your understanding of the differences of the
variances.

Distribution: Via figures and text, explain how one can tell the
distribution is approximately normal.

\end{document}
